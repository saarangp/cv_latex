\newenvironment{projectlong}[2]{%
  \begin{tabularx}{\linewidth}{@{}X r@{}}
    \textbf{#1} & #2 \\[3pt]
  \end{tabularx}
  \begin{minipage}[t]{0.85\linewidth} % adjust width (e.g., 0.85–0.9)
  \begin{itemize}[leftmargin=1em, label=--, topsep=2pt, itemsep=3pt]
}{%
  \end{itemize}
  \end{minipage}
  \vspace{0.6em} % <-- add or adjust this (e.g., 0.5em–1em)
}



\section*{Selected Research Projects}

\begin{projectlong}{EEG Foundation Model for Cross-Dataset Neural Representation Learning}{2024--Present}
    \item Developing an EEG foundation model based on JEPA that can learn reusable and transferable representations. 
    \item Trained on 30,000 hours and over 20K subjects and evaluated on both BCI and clinical tasks to assess cross-task and cross-cohort transfer.
    \item Built a cross-dataset evaluation suite covering transfer, robustness to channel mismatch, and low-label adaptation across research and clinical tasks.
\end{projectlong}


\begin{projectlong}{Epilepsy Dynamics and Biomarkers for Closed-Loop Neurostimulation}{2023--Present}
    \item Established evidence for thalamic involvement in seizure dynamics in pediatric epilepsy using SEEG spectral and connectivity analyses.
    \item Characterized thalamocortical interaction patterns during seizures to clarify candidate mechanisms of modulation and termination.
    \item Investigating biomarker-driven deep learning approaches for identifying adaptive closed-loop thalamic stimulation protocols in pediatric epilepsy.
\end{projectlong}
